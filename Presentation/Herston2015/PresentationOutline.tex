\documentclass[11pt]{article}
\usepackage{verbatim}
%\usepackage{html}
\usepackage{wasysym}
\usepackage{natbib}
\usepackage{graphicx}
\usepackage{ifpdf}
\ifpdf
\DeclareGraphicsExtensions{.pdf,.png,.jpg}
\else
\DeclareGraphicsExtensions{.eps}
\fi

\usepackage{setspace}
\pagestyle{myheadings}
\markright{Manuscript outline}

%\setlength{\oddsidemargin}{.1375in}
%\setlength{\evensidemargin}{.1375in}
\setlength{\oddsidemargin}{.15cm}
\setlength{\evensidemargin}{.15cm}
\setlength{\textwidth}{16cm}
\setlength{\textheight}{24cm}

\setlength{\headsep}{0.9cm}
\setlength{\jot}{0.4cm}
\setlength{\topsep}{0.6cm}

\flushbottom

\long\def\symbolfootnote[#1]#2{\begingroup%
\def\thefootnote{\fnsymbol{footnote}}\footnote[#1]{#2}\endgroup} 

\long\def\symbolfootnotemark[#1]{\begingroup%
\def\thefootnote{\fnsymbol{footnote}}\footnotemark[#1]\endgroup} 

\thispagestyle{empty}% Remove page numbering

\begin{document}

\title{Hazard function models to estimate mortality rates affecting fishes}

\maketitle

PURPOSE: present an application of survival analysis to estimate mortality rates in fisheries / gather some advices / find collaborators

\tableofcontents

%%%%%%%%%%%%%%%%%%%%%%%%%%%%%%%%%%%%%%%%%%%%%%%%%%%%%%%%%%%%%%%%%%%%%%%%%%%%%%%%%%%%%%%%%%%%%%%%%%%%%%%%%%%%%%%%%%%%%%%%%%%%%%%%%%%%%%%%%%%%%%%%%%%%%%%%%%

\section{Slide 1: Fisheries stock assessment: what is it ?}

First of all, I'd like to thanks the organisers to invite me to talk about our research. I would like to take this opportunity to present our recent work on applying survival analysis to fisheries research. I hope that this example will highlight areas of work interesting to some of you and potentially develop collaborations.\\
I am Marco Kienzle, working as a biometrician for the department of agriculture of the government of Queensland.

%% \section{Slide 2: Why fisheries research ?}
%% \begin{itemize}
%% \item Australian Bureau of Agricultural and Resource Economics and Sciences (ABARES).
%% \item By comparison, meat production in Australia is worth 3 times more (\textasciitilde 7 billion AU\$) (ABARES - Australian Commodity Statistics 2013) and employs between 48.000--65.000 people. (MLA)
%% %\item As a comparison, Bunnings's revenue is 54 billion AU$ and employs \textasciitilde 36.000 people
%% \item Aquaculture production: the blue revolution
%% \end{itemize}

%% \section{Slide 3: Why calling for closer collaboration with statisticians ?}

%% \section{Slide 4: Moreton Bay trawl fishery}

%% \begin{itemize}
%% \item This fisheries catch complex mixture of species. The graph omitted both fish, elasmobranch and by-catch.
%% \item Data quality in fisheries can be poor sometimes. In this case we have no indication of major mis-reporting or tempering of the logbooks.
%% \end{itemize}

%% \section{Slide 5: Tiger prawn distribution and Moreton Bay fishery}
%%         \begin{itemize}
%%           \item Penaeus esculentus is a tropical species found in the north of the Australian continent
%%           \item the limits of its current distribution are in the sub-tropic
%%           \item Moreton Bay is the southern most commercial fishery known on the east coast although we are starting to have anecdotal evidence of tiger prawn catch in Northern New South Wales in the Clarence river
%%           \item Moreton Bay covers an area of 1500 km$^{2}$
%%           \item about 1/2 the area (850 km$^{2}$) is used as trawling grounds for prawns
%%           \item Queensland government closed some area to trawling in 2009 under the Moreton Bay Marine Park Plan.
%%           \end{itemize}

%% \section{Slide 6: Recent trends in tiger prawn price, fishing effort and catch}
%%         \begin{itemize}
%%           \item Australian Bureau of Statistics (ABS): prawn prices corrected by Consumer Price Index (CPI) \\
%%             $\rightarrow$ Price fell starting from 2000
%%           \item Total number of vessels/effort plummeted also from 2000 \\
%%            $\rightarrow$ strong correlation with price ($\rho = 0.79$)
%%           \item Meanwhile catch increased as well as catch per unit effort
%%             $\rightarrow$ suggesting that tiger prawn abundance responded positively to decreasing fishing effort\\
%%             $\rightarrow$ this is un-usual, in most places around the world we see declining abudance
%%             $\rightarrow$ Further modelling was pursued to account for technological improvement.
%%           \end{itemize}

%% \section{Slide 7: Mathematical and statistical models}

%% \begin{itemize}
%% \item A recursive equation
%% \item Invented in late 70s--early 80s, a approximated linearization to facilitate the calculations
%% \item Model is fitted to catch 
%% \item A likelihood approach to allow for model comparison
%% \end{itemize}

%% \section{Slide 8: Computing environment for implementation}
%% \begin{itemize}
%% \item The choice of C++ was in part guided by the availability of MINUIT
%% \item MINUIT: cern favourite optimization library written in C++
%% \end{itemize}

%% \section{Slide 9: Hypothesis 1: effect of technological creep}
%% \begin{itemize}
%% \item account for vessel size, engine power, net dimension, electronic technology
%% \item Used the Prawn Trawl Performance Model (PTPM) to combine all vessel and gear characteristics in a configuration specific swept area rate (in hectares/hour)
%% \item Increase of fishing power through time is estimated with a GLM
%% \end{itemize}

%% \section{Slide 10: Hypothesis 2: change in fisherman's preference}
%% \begin{itemize}
%% \item The definition of targeting was a major problem
%% \item The research group came up with literally dozen of definitions of tiger prawn targeting (largest catch, more than 50\% of catch, largest value, etc...)
%% \item The problem was solved with an ANCOVA taking the definition that minimized the residuals sum of square\\
%% THE IDEA BEING non-targeted effort catch tiger at random while targeted has to show a positive relationship between catch and effort\\
%% This translate into log(catch) is linearily related to log(effort) with a smaller slope (and null) for non-targeted events and a linear relationship with larger slope for targeted events
%% \end{itemize} 


%% \section{Slide 11: Hypothesis 3: environment effect}
%% \begin{itemize}
%%   \item Before I move onto describing the different models, I need to talk a little bit about the ecology of tiger prawn
%%   \item Old idea in the literature that temperature plays an important role on the dynamic of tiger prawn fisheries
%% \item Hill (1985) experimental data showed that tiger prawn were more active at higher temperature --> assumed to be related to physiology of these poikilotherms, more precisely that at lower temperature, digestion rates are lower and therefore the needs to feed reduced.

%% \end{itemize}

%% \section{Slide 12: Comparison of multiple fishing mortality models}

%% \section{Slide 13: Results}

%% \section{Slide 14: More results}

%% \section{Slide 15: Is tiger prawn over-exploited in Moreton Bay ?}
%% \begin{itemize}
%% \item Simulations showed that MSY = 145 $\pm$ 53 tonnes, Emsy = 5400 days (1990 units) = 4000 days (2010 units)
%% \item Large variability around the mean: environmental factors can substantially improve/reduce prawn production
%% \item Observed variability is well within model boundaries
%% \end{itemize}

%% \section{Slide 16: Conclusions}
%% \begin{itemize}
%% \item This stock moved from over-exploitation to under-exploitation driven falling prawn prices: the rise of aquaculture has reduced over-exploitation more effectively than any prior management arrangement
%% \end{itemize}

%% \section{Slide 17: Future research direction and opportunities for collaborations}

\end{document}

%%%%%%%%%%%%%%%%%%%%%%%%%%%%%%%%%%%%%%%%%%%%%%%%%%%%%%%%%%%%%%%%%%%%%%%%%%%%%%%%%%%%%%%%%%%%%%%%%%%%%%%%%%%%%%%%%%%%%%%%%%%%%%%%%%%%%%%%%%%%%%%%%%%%%%%%%%%%
