The exponential decrease in abundance of individuals belonging to a single cohort due to constant natural ($M$) and fishing ($F$) mortalities was described from a survival analysis point of view \citep{cox84b} using a constant hazard function of time ($t$) and parameters $\theta$

\begin{equation}
h(t; \theta) = M + F
\end{equation}

The probability density function (pdf) describing survival from natural and fishing mortality is

\begin{equation}
f(t; \theta) = (M + F) \ e^{-(M+F)t} = \underbrace{M \times e^{-(M+F)t}}_{=f_{1}(t; \theta)} + \underbrace{F \times e^{-(M+F)t}}_{=f_{2}(t; \theta)}
\end{equation}

Since age data belonging to individuals dying from natural causes are generally not available to fisheries scientists, we used only the component of the pdf that relates to fishing mortality ($f_{2}(t; \theta)$). This component of $f(t; \theta)$ integrates over the entire range of $t$ to 

\begin{eqnarray}
\int_{t=0}^{t=\infty} f_{2}(t; \theta) \ dt &=& \int_{t=0}^{t=\infty} F \times e^{-(M+F)t} \ dt\\
                                        &=& \int_{t=0}^{t=\infty} f(t; \theta) \ dt - \int_{t=0}^{t=\infty} M \times e^{-(M+F)t} \ dt\\
                                        &=& 1 - \int_{t=0}^{t=\infty} M \times e^{-(M+F)t} \ dt \\
                                        &=& 1 - \frac{M}{M+F}
\end{eqnarray}

Hence, the pdf of catch at age data was obtained by normalizing $f_{2}(t; \theta)$

\begin{eqnarray}
g(t; \theta) &=& \frac{1}{1 - \frac{M}{M+F}} \ f_{2}(t; \theta) \\
             &=& \frac{M+F}{F} \ F \times e^{-(M+F)t} \\
             &=& f(t; \theta)
\end{eqnarray} 

Following Fisher's definition \citep{edwards1992likelihood}, the likelihood of a sample of fish caught in the fishery ($S_{i}$) was written as 
%% \begin{equation}
%% \mathcal{L} = \prod_{i=1}^{n} \bigl ( \int_{t=a_{i}}^{t=a_{i+1}} g(t; \theta) \bigr ) ^ {a_{i}}
%% \end{equation}

\begin{eqnarray}
\mathcal{L}  &=& \prod_{i=1}^{n} \bigl ( \int_{t=a_{i}}^{t=a_{i+1}} f(t; \theta) \ dt \bigr ) ^ {S_{i}} \\
            &=& \prod_{i=1}^{n} P_{i} ^ {S_{i}}
\end{eqnarray}

This is often refered to as the likelihood of a multinomial probability ($P_{i}$) where $P_{i}=\int_{t=a_{i}}^{t=a_{i+1}} f(t; \theta) \ dt$.\\






The logarithm of the likelihood was
%% \begin{eqnarray}
%% {\rm log}(\mathcal{L}) &=& \sum_{i=1}^{n} a_{i} \ {\rm log} \bigl ( \int_{t=a_{i}}^{t=a_{i+1}} g(t; \theta) \bigr ) \\
%%                        &=& \sum_{i=1}^{n} a_{i} \ {\rm log} \bigl ( \int_{t=a_{i}}^{t=a_{i+1}} \frac{f_{2}(t; \theta)}{1 - \frac{M}{M+F}} \bigr ) \\
%%                        &=& \frac{1}{1 - \frac{M}{M+F}} \sum_{i=1}^{n} a_{i} \ {\rm log} \bigl ( \int_{t=a_{i}}^{t=a_{i+1}} f_{2}(t; \theta) \bigr ) \\
%%                        &=& \frac{1}{1 - \frac{M}{M+F}} \sum_{i=1}^{n} a_{i} \ {\rm log} \bigl ( \frac{F}{M+F} (e^{-(M+F) \times a_{i}} - e^{-(M+F) \times a_{i+1}}) \bigr ) \\
%% \end{eqnarray}

\begin{eqnarray}
{\rm log}(\mathcal{L}) &=& \sum_{i=1}^{n} S_{i} \ {\rm log} \bigl ( \int_{t=a_{i}}^{t=a_{i+1}} f(t; \theta) \ dt \bigr ) \\
                       &=& \sum_{i=1}^{n} S_{i} \ {\rm log} \bigl ( \int_{t=a_{i}}^{t=a_{i+1}} (M + F) \ e^{-(M+F)t} \ dt \bigr ) \\
                       &=& \sum_{i=1}^{n} S_{i} \ {\rm log} \bigl ( e^{-(M+F) \times a_{i}} - e^{-(M+F) \times a_{i+1}} \bigr ) \\
%                       &=& \sum_{i=1}^{n} S_{i} \ {\rm log} \bigl ( e^{-(M+F) \times a_{i}} (1 - e^{-(M+F)}) \bigr ) \\ % this is not correct because $a_{i+1} \neq a_{i}+1$
\end{eqnarray}

This development illustrated an application of survival analysis to estimate mortality rates affecting a cohort of fish by maximum-likelihood using a sample of catch at age. This method was implemented in R \citep{R} in the package Survival Analysis for Fisheries Research (SAFR) provided as supplement material. Numerical application were made available using the following commands: {\bf library(SAFR); example(llfunc1)}.\\

Natural and fishing mortality cannot be disentangled with catch data only but the next section will show that the provision of effort data allowed to estimate both catchability ($q$) and natural mortality. 
