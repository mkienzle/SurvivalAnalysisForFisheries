The exponential decrease in abundance of individuals belonging to a single cohort due to constant natural ($M$) and fishing ($F$) mortality rates was described from a survival analysis point of view \citep{scimar42,cox84b} using a constant hazard function of time ($t$) and parameters $\theta$

\begin{equation}
h(t; \theta) = M + F
\end{equation}

The probability density function (pdf) describing survival from natural and fishing mortality is

\begin{equation}
f(t; \theta) = (M + F) \ e^{-(M+F)t} = \underbrace{M \times e^{-(M+F)t}}_{=f_{1}(t; \theta)} + \underbrace{F \times e^{-(M+F)t}}_{=f_{2}(t; \theta)}
\end{equation}

Since age data belonging to individuals dying from natural causes are generally not available to fisheries scientists, we used only the component of the pdf that relates to fishing mortality ($f_{2}(t; \theta)$). This component of $f(t; \theta)$ integrates over the entire range of $t$ to 

\begin{align}
 \int_{t=0}^{t=\infty} f_{2}(t; \theta) \ dt &= \int_{t=0}^{t=\infty} F \times e^{-(M+F)t} \ dt\\
                                         &= \int_{t=0}^{t=\infty} f(t; \theta) \ dt - \int_{t=0}^{t=\infty} M \times e^{-(M+F)t} \ dt\\
                                         &= 1 - \int_{t=0}^{t=\infty} M \times e^{-(M+F)t} \ dt \\
                                         &= 1 - \frac{M}{M+F}
\end{align}

Hence, the pdf of catch at age data was obtained by normalizing $f_{2}(t; \theta)$

\begin{eqnarray}
g(t; \theta) &=& \frac{1}{1 - \frac{M}{M+F}} \ f_{2}(t; \theta) \\
             &=& \frac{M+F}{F} \ F \times e^{-(M+F)t} \\
             &=& f(t; \theta)
\end{eqnarray} 

The likelihood \citep{edwards1992likelihood} of a sample of fish caught in the fishery ($S_{i}$) was written as 

\begin{eqnarray}
\mathcal{L}  &=& \prod_{i=1}^{n} \bigl ( \int_{t=a_{i}}^{t=a_{i+1}} f(t; \theta) \ dt \bigr ) ^ {S_{i}} \\
            &=& \prod_{i=1}^{n} P_{i} ^ {S_{i}}
\end{eqnarray}

%This is often referred to as the likelihood of a multinomial probability ($P_{i}$) where 

%\begin{eqnarray}
% P_{i} &=& \int_{t=a_{i}}^{t=a_{i+1}} f(t; \theta) \ dt \\
%      &=& e^{[-(M+F) a_{i}]} \bigl ( 1 - e^{[- (M+F) \times (a_{i+1} - a_{i})]}  \bigr )
%\end{eqnarray}

\noindent where $P_{i}$ is the probability of dying in the interval $[a_{i}; a_{i+1}]$.\\

The logarithm of the likelihood was
\begin{eqnarray}
{\rm log}(\mathcal{L}) &=& \sum_{i=1}^{n} S_{i} \ {\rm log} \bigl ( \int_{t=a_{i}}^{t=a_{i+1}} f(t; \theta) \ dt \bigr ) \\
                       &=& \sum_{i=1}^{n} S_{i} \ {\rm log} \bigl ( \int_{t=a_{i}}^{t=a_{i+1}} (M + F) \ e^{-(M+F)t} \ dt \bigr ) \\
                       &=& \sum_{i=1}^{n} S_{i} \ {\rm log} \bigl ( e^{-(M+F) \times a_{i}} - e^{-(M+F) \times a_{i+1}} \bigr )
\label{eq:16}
\end{eqnarray}

The log-likelihood can accommodate a last age-group made of all observations above a certain age in the sample (referred to as a +group) as follow \citep{pawitan2013all}

\begin{equation}
{\rm log}(\mathcal{L}) = \sum_{i=1}^{n-1} S_{i} \ {\rm log} \bigl ( e^{-(M+F) \times a_{i}} - e^{-(M+F) \times a_{i+1}} \bigr ) + S_{n} \ {\rm log} \bigl ( e^{-(M+F) \times a_{n}} \bigr )
\end{equation}

%% In cases where sampling does not cover the entire range of age ($ i \notin in [1;n], i \in [a;b]$ with $a > 1$ and $b < n$) for example because younger fish live outside the sampling area, the distribution is truncated ($\sum_{i} P_i < 1$) and using Eq.~\ref{eq:16} would lead to erroneous estimations. In such cases, the relative proportions $p_i=P_{i} / \sum_{i=a}^{b} P_{i}$ are to be used in the likelihood function. Conditional on the total sample $S$ with age between age $a_i$ and $a_{i+1}$, with $i \in [a;b]$, the age frequency of the total sample $S$ follows the following multinominal distribution (up to a constant $S!/S_a! \ ... \ S_b!$) (see \cite{Wang99a})

%% \begin{equation}
%% \mathcal{L} = \prod_{i=a}^{b} p_{i} ^ {S_{i}}.
%% \end{equation}

This development illustrated an application of survival analysis to estimate mortality rates affecting a cohort of fish by maximum-likelihood using a sample of catch at age. This method was implemented in R \citep{R} in the package Survival Analysis for Fisheries Research (SAFR) available at \url{https://github.com/mkienzle/SurvivalAnalysisForFisheries}. \\ %provided as supplement material. Numerical application were made available using the following commands: {\bf library(SAFR); example(llfunc1)}.\\

Natural and fishing mortality cannot be disentangled with catch data only but the next section will show that the provision of effort data allowed to estimate both catchability ($q$) and natural mortality. 
