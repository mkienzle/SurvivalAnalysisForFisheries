\documentclass[11pt]{article}
\usepackage{verbatim}
%\usepackage{html}
\usepackage{graphicx}
\usepackage{setspace}
\pagestyle{myheadings}
\markright{Manuscript submission}

%\setlength{\oddsidemargin}{.1375in}
%\setlength{\evensidemargin}{.1375in}
\setlength{\oddsidemargin}{.15cm}
\setlength{\evensidemargin}{.15cm}
\setlength{\textwidth}{16cm}
\setlength{\textheight}{24cm}

\setlength{\headsep}{0.9cm}
\setlength{\jot}{0.4cm}
\setlength{\topsep}{0.6cm}

\flushbottom

\long\def\symbolfootnote[#1]#2{\begingroup%
\def\thefootnote{\fnsymbol{footnote}}\footnote[#1]{#2}\endgroup} 

\long\def\symbolfootnotemark[#1]{\begingroup%
\def\thefootnote{\fnsymbol{footnote}}\footnotemark[#1]\endgroup} 

\thispagestyle{empty}% Remove page numbering

\begin{document}

\begin{spacing}{1.2} 
\hspace{9cm} Brisbane, on the 13$^{th}$ January 2015\\ \\

\noindent Dear Canadian Journal of Fisheries and Aquatic Sciences editor, \\ 

%\newline
%\newline
\noindent We are writing to your journal to submit for publication, as a full paper, a manuscript entitled {\bf Maximum likelihood estimates of mortality rates from catch at age data using survival analysis} that has not been published or simultaneously submitted for publication elsewhere. This article describes a novel method to create likelihood functions of age data for fisheries research. This research was motivated by the necessity to develop methods to process a large number of datasets collected by the long-term monitoring program of the government of Queensland (Australia). The method development originated in 2002 from the realization of the existence of the statistical counterpart to the exponential model. This approach was refined over the years by successive discussions with statisticians in Europe and Australia and fully matured by embracing the body of research generated by the field of statistics called survival analysis. \\

Applying survival analysis concepts and methods to age data collected in fisheries research has open a new way to look at an old problem. It allowed to estimate natural mortality for Sea Mullet, a parameter we thought previously was not possible to estimate due to lack of information. This new method has improved our efficiency of processing age datasets. We believe that this application could be a burgeoning field of research in fisheries. Hence, we decided to submit this manuscript to involve our research community into developments and applications of survival analysis to stock assessment. \\

In our opinion, the best researcher to review this paper would have a mixed background with experience in both survival analysis and fisheries stock assessment. Our colleagues in applied statistics for medical research who have read a copy of the manuscript have had no problem with the application of survival analysis concepts to age data in fisheries but failed to appreciate the subtleties of fisheries data. On the other hand, our colleagues in stock assessment were often overwhelmed by the shift in perspective required to apply survival analysis instead of a more traditional approach, in particular the benefit of using hazard functions to devise mortality schedules and the associated probability distributions functions. In our opinion, the following researchers would provide a valuable review and suggestions to this manuscript:

\begin{table}[h!]
\small
%\begin{center}
\begin{tabular}{llllll} 
\hline
          & Dr A. Punt               & Dr W.N. Venables & Dr R.B. Millar          & Dr R.I.C.C. Francis \\
\hline
&&&& \\
  Address & Univ. of Washington      & CSIRO            & Univ. of Auckland       & NIWA \\ 
          &                          &                  & Private Bag 92019       & Private Bag 14901 \\
          & Seattle                  & Brisbane         & Auckland                & Wellington \\
          & U.S.A.                   & Australia        & New Zealand             & New Zealand \\

Phone     &                          & +61 7 3826 7251  &                         &              \\
Fax       & & & & \\
Email     & aepunt@u.washington.edu  & Bill.Venables@csiro.au & millar@stat.auckland.ac.nz & c.francis@niwa.co.nz \\
\hline
\end{tabular}
\end{table}

We wrote a companion library written in R, called Survival Analysis for Fisheries Research (SAFR), with this manuscript to demonstrate the use of concepts presented as well as providing an implementation of survival analysis methods described in the manuscript. Provision of this library to the reviewers is essential to demonstrate how to apply the theory exposed. The SAFR is available by e-mail request to Marco.Kienzle@daff.qld.gov.au.\\ % Our testing of this library found that the {\it optim} routine in R did not work on our 32-bit machines while it performed according to expectations on 64-bit systems. This glitch is beyond our control and we are awaiting a fix. Until then, we recommend you and your reviewers to run the examples from the SAFR library on 64-bit machines.\\ 


Given that this manuscript contains several long equations which formatting into two-columns layout was awkward, we decided to provide it formatted into a single column for clarity of presentation to you and your reviewers. In the event that you accept this manuscript for publication, we will reformat it to fit the standard style of your journal. We hope that you will find this work interesting and relevant to our scientific community. All co-authors fully participated and accept responsibility for this work. \\

\noindent Regards, \\
\noindent M. Kienzle, J. McGilray and Y.G. Wang

\end{spacing}

%\hline
\vspace{3cm}

\noindent Marco Kienzle\\
Dept of Agriculture, Fisheries and Forestry of the Govt of Queensland\\ 
Level A2, Ecosciences Precinct, Joe Baker St\\ 
Dutton Park  QLD-4102\\
Australia \\
Telephone 07 3255 4227 Fax 3846 1207\\ 
Email Marco.Kienzle@daff.qld.gov.au\\ 
%29 Kamarin st\\
%Manly-West QLD-4179\\
%Australia\\
%Email: Marco.Kienzle@gmail.com\\
%Phone: +61 4 05795090 \\ \\

\end{document}
