One purpose of stock assessment is to estimate mortality rates affecting fish stocks. This estimation problem is easier to solve for species that can be aged as opposed to those for which age can't be determined, as for example crustaceans. The reason is that mortality and longevity are inversely related, hence age is a measure of mortality. The central mortality model in fisheries research relating catch to the number of fish belonging to a cohort through time was proposed by Baranov \citep{quin99b}. Given recruitment and mortality rates, the proportions of individuals at age in the catch can be calculated and used in a multinomial likelihood  \citep{Four82a}. This method has become by far the most common likelihood to integrate age data into modern stock assessment models \citep{Francis201470, Maunder201361}.\\

The deterministic exponential model in Baranov's catch equation has a statistical counterpart in the form of the exponential probability distribution function which first and second moments quantify the relationship between longevity and mortality rate \citep{cow98b}: the mean age of a cohort which abundance declines at a constant rate is the inverse of that rate. Adopting such a statistical view of the exponential decay of individual belonging to a cohort allowed to develop a set of maximum likelihood functions to estimate parameters of importance when assessing stocks. The field of survival analysis in statistics has created both a conceptual framework and refined methods to estimate mortality rates \citep{kleinbaum2005survival,cox84b} which are widely applied in medical research and engineering. \\ 

Despite the commonalities between survival analysis for medical and fisheries research, this theory has seldom been applied to stock assessment and animal ecology \citep{scimar42, Pollock1989}. In this manuscript, we described how to apply survival analysis to create likelihood functions of age data for the purpose of estimating natural and fishing mortality rates as well as gear selectivity. We started with a simplistic example using constant natural and fishing mortality rates to introduce fundamental concepts from survival analysis before moving to more sophisticated cases leading to its application to real data from the sea mullet fishery in Queensland (Australia). The proposed methods were tested with simulated data to characterize some of their properties and their capacity to estimate population dynamic parameters of interest. Finally, the application to the mullet fishery case study provided specific estimates of natural mortality, catchability and selectivity. \\
