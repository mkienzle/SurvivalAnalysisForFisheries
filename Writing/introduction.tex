One of the purposes of stock assessment is to estimate mortalities affecting fish stocks. This task is much easier for species that can be aged compared to, for example crustaceans, which aging is not possible. The reason lies in that mortality and longevity are inversely related hence age is a measure, albeit inverse, of mortality. The central mortality model used in fisheries research was proposed by Baranov to describe the variation of the number of fish belonging to a cohort through time \citep{quin99b}. This deterministic exponential model has a statistical counterpart in the form of the exponential probability distribution function which first and second moments quantify the relationship between longevity (age) and mortality rate \citep{cow98b}. Adopting a statistical view of this problem allowed to develop maximum likelihood estimators \citep{Burnb03} of parameters of importance to stock assessment scientists. The branch of statistics focused on survival analysis has created and refined methods to estimate mortality rates \citep{cox84b} which are widely applied in the fields of medical research and engineering. \\ 

%The present article describes an application of survival analysis to fisheries catch at age data.
Despite the commonalities between survival analysis for medical and fisheries research, this theory has seldom been applied to animal ecology \citep{Pollock1989}: to our knowledge, there hasn't been any application to fish age data for the purpose of stock assessment. In this manuscript, we described how to apply survival analysis to create likelihood functions of catch at age for the purpose of estimating natural and fishing mortalities as well as gear selectivity. We started with a simplistic example of constant natural and fishing mortality to introduce fundamental concepts from survival analysis before moving to more sophisticated cases leading to its application to real data from the sea mullet fishery in Queensland (Australia). The proposed methods were tested with simulated data to characterize some of their properties and their capacity to estimate population dynamic parameters of interest. Finally, the application to the mullet fishery case study provided specific estimates of natural mortality, catchability and selectivity. \\
