Fisheries management agencies around the world collect age data for the purpose of assessing the status of natural resources in their jurisdiction. Estimates of mortality rates represent a key information to assess the sustainability of fish stocks exploitation. Contrary to medical research or manufacturing where survival analysis is routinely applied to estimate failure rates, survival analysis has seldom been applied in fisheries stock assessment despite similar purposes between these fields of applied statistics. In this paper, we developed hazard functions to model the dynamic of an exploited fish population. These functions were used to estimate all parameters necessary for stock assessment (including natural and fishing mortality rates as well as gear selectivity) by maximum likelihood using age data from a sample of catch. This novel application of survival analysis to fisheries stock assessment was tested by Monte Carlo simulations to assert that it provided un-biased estimations of relevant quantities. The method was applied to data from the Queensland (Australia) sea mullet ({\it Mugil cephalus}) commercial fishery collected between 2007 and 2014. It provided, for the first time, an estimate of natural mortality affecting this stock: 0.22 $\pm$ 0.08 year$^{-1}$.


