Fisheries management agencies around the world collect age data for the purpose of assessing the status of natural resources in their juridiction. Estimates of mortality rates are central to assess the sustainability of fish stocks exploitation. Survival analysis has seldom been applied to fisheries research despite its widespread use in medical research and engineering to estimate failure rates. In this paper, we present a variety of hazard functions to model the dynamic of a fishery and estimate by maximum likelihood all parameters necessary for a stock assessment (including natural and fishing mortality rates as well as gear selectivity) from a sample of fish age. These methods were tested by Monte Carlo simulations to assert that they provide un-biased estimates of these quantities. An application to the Queensland's sea mullet fishery (Australia) dataset provided an estimate of natural mortality equal to 0.319 $\pm$ 0.165 year$^{-1}$.


%% Survival analysis has seldomed be used in this area of research despite estimate fish mortality ratesSurvival analysis was applied to fisheries catch at age data to develop maximum likelihood estimators for stock assessment. This new method estimated natural mortality, fishing mortality and catchability from typical catch at age matrices. Monte Carlo simulations suggested estimates were unbiased and provided a better fit than the traditional multinomial approach. 

%% Application to a dataset from Queensland's sea mullet fishery (Australia) estimated natural mortality to be equal to 0.319 $\pm$ 0.165 year$^{-1}$.
