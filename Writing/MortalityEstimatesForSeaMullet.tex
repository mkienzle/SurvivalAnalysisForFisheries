Applying survival analysis to age data from a sample of Sea Mullet weighted by total yearly catch, catchability was estimated to be equal to 7.055 $\pm$ 2.724 $10^{-5}$ per boat-day (Tab.~\ref{tab:Sensitivity-CatchabilityToMulletDataTruncation}). Natural mortality for Sea Mullet was estimated to 0.319 $\pm$ 0.165 year$^{-1}$ using the entire dataset (comprising 2013 and 16 age-groups, Tab.~\ref{tab:Sensitivity-NaturalMortalityToMulletDataTruncation}). The sensitivity analysis showed consistent estimates with the removal of 1 or 2 years and up to 6 age-groups: catchability estimates varied between [7.054; 7.126] 10$^{-5}$ with mean equal to 7.079 $10^{-5}$ boat.days$^{-1}$ and natural mortality estimates varied between [0.319; 0.382] with mean equal to 0.336 year$^{-1}$. This sensitivity analysis suggested that the presence of age-groups in the dataset with fewer, sparse observations increased the uncertainty of both catchability and natural mortality estimates.\\

The maximum likelihood matrix of probabilities ($P_{i,j}$) associated with the weighted observations at age in the sample ($S^{*}_{i,j}$) were presented in Tab.~\ref{tab:MaximumLikelihoodProbabilitiesOfMulletAgeSampleWeightedByTotalCatch}. They illustrate that the construction of the likelihood estimator using this survival analysis relied on probabilities summing to 1 along the cohort instead of summing to one along rows and across cohorts, as previously proposed to develop the multinomial likelihood of age data by \cite{Four82a}. Note that 2 cohorts in the dataset were described by a single observation (top-right and bottom-left corner of the matrix in Tab.~\ref{tab:MaximumLikelihoodProbabilitiesOfMulletAgeSampleWeightedByTotalCatch}) which did not provide any information to estimate mortality rates, as represented by their associated probability equal to 1.\\

Maximum likelihood estimates of gear selectivity, catchability and natural mortality were slightly affected by weighting the sample of observed number at age by total yearly catch (Tab.~\ref{tab:EffectOfWeightingOnMulletEstimates}), suggesting that variation of catch within $\pm$ 12\% of the coefficient of variation influenced on the outcome of the analysis (Tab.~\ref{tab:Mullet-NbAtAge}).

