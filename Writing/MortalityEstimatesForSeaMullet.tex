Sea Mullet data showed larger catch per unit of effort in the ocean than in estuaries (Tab.~\ref{tab:Mullet-NbAtAge}). Of all three models compared with AIC, the model that assumed catchability varied between habitats and selectivity was the same in both habitats (model 2) was best supported by the data (Tab.~\ref{tab:MulletModelComparison}). This model estimated catchability in the ocean to be 16 times larger than in estuaries (Tab.~\ref{tab:BestParameterEstimates}). Natural mortality for Sea Mullet was estimated to be equal to 0.219 $\pm$ 0.082 year$^{-1}$. Estimates of gear selectivity suggested it increased up to the fifth age-group, beyond which fishes were fully selected by the fishing gear.\\

The maximum likelihood matrix of probabilities ($P_{i,j}$) associated with the weighted observations at age in the sample ($S^{*}_{i,j}$) were presented in Tab.~\ref{tab:MaximumLikelihoodProbabilitiesOfMulletAgeSampleWeightedByTotalCatch}. They illustrate that the construction of the likelihood estimator using survival analysis created probabilities summing to 1 along each cohort instead of summing to one along years (rows) and across cohorts, as previously proposed to develop the multinomial likelihood of age data by \cite{Four82a}. Note that 2 cohorts in the dataset were described by a single observation (top-right and bottom-left corner of the matrix in Tab.~\ref{tab:MaximumLikelihoodProbabilitiesOfMulletAgeSampleWeightedByTotalCatch}) which did not provide any information to estimate mortality rates, as represented by their associated probability equal to 1.\\



