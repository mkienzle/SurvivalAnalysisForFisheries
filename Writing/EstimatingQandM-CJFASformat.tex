In this section, we assumed that a time series of effort ($E_{i}$) associated with a sample of catch at age ($S_{i}$) was available to the researcher. And the assumption that fishing mortality varied according to fishing effort through constant catchability ($q$) held: $F(t) = q \ E(t)$. In this situation, the hazard function was written as

\begin{equation}
h(t,\theta) = M + q \ E(t)
\end{equation} 

And the pdf

\begin{align}
& f(t, \theta) = ( M + q \ E(t)) \ e^{-Mt-q\int_{0}^{t}E(t) \ dt} \\
             &= \underbrace{M \times e^{-Mt-q\int_{0}^{t}E(t) \ dt}}_{=f_{1}(t; \theta)} + \underbrace{q \ E(t) \times e^{-Mt-q\int_{0}^{t}E(t) \ dt}}_{=f_{2}(t; \theta)}
\end{align}

As in the previous section, we had

\begin{align}
& \int_{t=0}^{t=\infty} f_{2}(t; \theta) \ dt = \\
& 1 - \int_{t=0}^{t=\infty} M \times e^{-Mt-q\int_{0}^{t}E(t) \ dt} \ dt \\
\end{align}

But we did not know an analytic solution to the integral since the function $E(t)$ was not specified. Nevertheless, as we knew the value of effort in any given interval ($\int_{t=a_{i}}^{t=a_{i+1}}  E(t) \ dt = E_{i} = \int_{t=0}^{t=a_{i+1}} E(t) \ dt - \int_{t=0}^{t=a_{i}} E(t) \ dt, \forall i \in [1; n]$), we could calculate the value of $\int_{t=0}^{t=\infty} f_{2}(t; \theta) \ dt$ assuming $E(t)$ was constant over each interval $i$ 

\begin{align}
& \int_{t=0}^{t=\infty} f_{2}(t; \theta) = \\
& 1 - \sum_{i=1}^{n} \bigl [ -\frac{M}{M+q \ E_{i}} e^{-Mt-q\int_{0}^{t}E(t) \ dt} \bigr ]_{t=a_{i}}^{t=a_{i+1}} = \\
& 1 - \sum_{i=1}^{n} \frac{M}{M+q \ E_{i}} \bigl ( e^{-M \ a_{i}-q \int_{0}^{a_{i}}E(t) \ dt} - e^{-M \ a_{i+1}-q \int_{0}^{a_{i+1}} E(t) \ dt} \bigr ) = \\
& \sum_{i=1}^{n} \frac{q \ E_{i}}{M+q \ E_{i}} \bigl ( e^{-M \ a_{i}-q\int_{0}^{a_{i}}E(t) \ dt} - e^{-M \ a_{i+1}-q\int_{0}^{a_{i+1}}E(t) \ dt} \bigr )
\end{align}

In practice, $ 0 \leq \int_{t=0}^{t=\infty} f_{2}(t; \theta) \leq 1$ and took a specific value depending on the values of $M, q $ and $E_{i}$. Naming this constant value $K$, we could write the pdf of catch at age given that effort data are available as

\begin{equation}
g(t; \theta) = \frac{1}{K} \ f_{2}(t; \theta)
\end{equation}

And the log-likelihood:
\begin{equation}
{\rm log}(\mathcal{L}) = \sum_{i=1}^{n} S_{i} \ {\rm log} \bigl ( \int_{t=a_{i}}^{t=a_{i+1}} g(t; \theta) \ dt \bigr )
\end{equation}

Numerical application of this method were made available using the following commands: {\bf library(SAFR); example(llfunc2)}.\\

Accounting for age-specific gear selectivity ($s(t)$) effects on fishing mortality ($F(t) = q \ s(t) \ E(t)$) was included in a similar way into the likelihood using constant value for selectivity at age. In practice, it is difficult to estimate $n$ additional selectivity parameters using only the data from a single cohort but processing several cohorts at the same time and assuming separability of fishing mortality rendered estimation of catchability, natural mortality and selectivity possible.
