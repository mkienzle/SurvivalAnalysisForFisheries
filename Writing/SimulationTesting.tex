Methods to estimate mortality and selectivity from a matrix containing a sample of number at age were tested with simulated datasets to characterize their performance. Variable number of cohorts ($n+p-1 = 25$, 35 or 45); maximum age ($p=8$, 12 or 16 years) and sample size of age measurement in each year varying from 125 to 2000 increasing successively by a factor 2. The simulated datasets were created by generating an age-structure population using random recruitment for each cohort, random constant natural mortality, random catchability and random fishing effort in each year (Tab.~\ref{tab:SimulationParameters.tex}). A catch at age matrix was calculated using a logistic gear selectivity with 2 parameters: 

\begin{equation}
s_{a_{i}} = \frac{1}{1  + {\rm exp}( \alpha - \beta \times a_{i})}
\end{equation}

Several sampling strategies were implemented to assess how it affected mortality estimates. To test estimators derived from survival analysis, one would like to draw randomly from the probability distribution. This is obviously impossible in the real world because field biologists never have in front of them a entire cohort to chose from. Nevertheless, we implemented a sampling strategy (sampling strategy 1) that randomly selected from the entire simulated catch at age dataset as a benchmark. In the real world, samples can be drawn by accessing only a single year-class of every cohort every year, so the second strategy implemented was to simulate a random selection of a fixed number of sample ($N$) each year (sampling strategy 2). Finally, the third strategy investigated was to apply a weighting by the estimated total catch at age ($\hat{C}_{i,j}$) to the sample of number at age in the sample ($S_{i,j}$) -- sampling strategy with weighting :

\begin{equation}
\hat{C}_{i,j} = p_{i,j} \odot C_{i} \otimes v(j)
\end{equation}

\noindent where $p_{i,j}$ is the proportion at age (see Appendix p.~\pageref{Appendix:DefinitionsOfMathematicalSymbols}), $C_{i}$ is a column vector containing the total number of fish caught in each year $i$ and $v(j)$ is a row vector of 1's. A weighted sample ($S^{*}_{i,j}$) was obtained using the fraction of total catch sampled
\begin{equation}
S^{*}_{i,j} = \hat{C}_{i,j} \times \frac{\sum_{i,j} S_{i,j}}{\sum_{i} C_{i}}
\end{equation}

Note that $\sum_{i,j} S_{i,j} = \sum_{i,j} S^{*}_{i,j}$.\\
%implemented a strategy that collected a number of samples in proportion to fishing effort (SS3).\\

%A sample of $N$ individuals per year was drawn at random from this matrix of catch at age, creating a matrix of sampled number at age data, with dimensions $n \times p$ containing $n \times N$ data, that were processed with survival analysis methods described in previous sections.\\ %Multiplicative white noise of magnitude 0\%, $\pm$ 10\%, ..., $\pm$ 50\% was applied to these simulated data to assess how it influenced uncertainty on parameter estimates.

Comparisons with the multinomial likelihood proposed by \cite{Four82a} were made using differences in negative log-likelihood between that method and the survival analysis approach described in the present article. Simulated catch were used to calculate the proportion of individual at age, constraining them to sum to 1 in each year. This method to calculate proportions for the multinomial likelihood was regarded as the best case scenario because we expect any estimation algorithm based on the multinomial likelihood to, at best, match exactly the simulated catch at age. The logarithm of these proportions were then multiplied by the simulated age sample (weighted or not depending on the case) to calculate the log-likelihood as described in \cite{Four82a}. This quantity was compared to that calculated using the survival analysis approach to determine which model best fitted the simulated data. This comparison ignored the number of parameters used in each model as the Akaike criteria would. The multinomial likelihood requires $n+p-1$ more parameters to be estimated than the survival analysis because the former requires an estimate of recruitment for each cohort in order to calculate the proportion at age in the catch.
