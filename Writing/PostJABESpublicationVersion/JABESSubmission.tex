\documentclass[11pt]{article}
\usepackage{verbatim}
%\usepackage{html}
\usepackage{graphicx}
\usepackage{setspace}
\pagestyle{myheadings}
\markright{Manuscript submission}

%\setlength{\oddsidemargin}{.1375in}
%\setlength{\evensidemargin}{.1375in}
\setlength{\oddsidemargin}{.15cm}
\setlength{\evensidemargin}{.15cm}
\setlength{\textwidth}{16cm}
\setlength{\textheight}{24cm}

\setlength{\headsep}{0.9cm}
\setlength{\jot}{0.4cm}
\setlength{\topsep}{0.6cm}

\flushbottom

\long\def\symbolfootnote[#1]#2{\begingroup%
\def\thefootnote{\fnsymbol{footnote}}\footnote[#1]{#2}\endgroup} 

\long\def\symbolfootnotemark[#1]{\begingroup%
\def\thefootnote{\fnsymbol{footnote}}\footnotemark[#1]\endgroup} 

\thispagestyle{empty}% Remove page numbering

\begin{document}

\begin{spacing}{1.2} 
\hspace{9cm} Brisbane, on the 3$^{rd}$ June 2015\\ \\

\noindent Dear editor of the Journal of Agricultural, Biological, and Environmental Statistics, \\

%\newline
%\newline
\noindent We are writing to your journal to submit for publication, as a full paper, a manuscript entitled {\bf Hazard function models to estimate mortality rates affecting fish populations with application to the sea mullet ({\it Mugil cephalus}) fishery on the Queensland coast (Australia)} that has not been published or simultaneously submitted for publication elsewhere. This article describes a novel method to create likelihood functions of age data collected from fisheries catch. This work was motivated by the necessity to develop methods to process a large number of data sets collected by the long-term monitoring program of the government of Queensland (Australia). The method development originated in 2002 from the realization of the existence of the statistical counterpart to the exponential model. This approach was refined over the years by successive discussions with statisticians in Europe and Australia and fully matured by embracing the body of research generated by survival analysis. \\

Applying survival analysis concepts and methods to age data collected in fisheries research has open a new way to look at an old problem. It allowed to estimate natural mortality for sea mullet, a parameter we thought previously was not possible to estimate due to lack of information. This new method has improved our efficiency of processing age data sets. We believe that this application could be a burgeoning field of application of survival analysis to fisheries research. \\%Hence, we decided to submit this manuscript to involve quickly our research community into developments and applications of survival analysis to stock assessment. \\

In our opinion, the best researchers to review this paper would have a background in applied statistics, in particular survival analysis. Our colleagues working in this area for medical research who have read a copy of the manuscript have had no problem with the concepts of applying survival analysis to fisheries age data. \\%, only a slight unfamiliarity with fisheries jargon.
% but failed to appreciate the subtleties of fisheries data. On the other hand, our colleagues in stock assessment were often overwhelmed by the shift in perspective required to apply survival analysis instead of a more traditional approach, in particular the benefit of using hazard functions to devise mortality schedules and the associated probability distributions functions. In our opinion, the following researchers would provide a valuable review and suggestions to this manuscript:

%% \begin{table}[h!]
%% %\begin{center}
%% \begin{tabular}{lllll} 
%% \hline
%%           & Dr A. Punt               & Dr W.N. Venables & Dr R.I.C.C. Francis \\
%% \hline
%% &&&& \\
%%   Address & Univ. of Washington      & CSIRO            & NIWA \\ 
%%           &                          &                  & Private Bag 14901 \\
%%           & Seattle                  & Brisbane         & Wellington \\
%%           & U.S.A.                   & Australia        & New Zealand \\

%% Phone     &                          & +61 7 3826 7251  &              \\
%% Fax       & & & & \\
%% Email     & aepunt@u.washington.edu  & Bill.Venables@csiro.au & c.francis@niwa.co.nz \\
%% \hline
%% \end{tabular}
%% \end{table}


%We understand that this manuscript is not perfectly formatted according to the rules required by your journal and provided in the author's guide. We will correct this manuscript accordingly in the event it is accepted for publication. 

We provided a companion library written in R, called Survival Analysis For R (SAFR), with this manuscript to demonstrate the use of concepts presented as well as providing an implementation of survival analysis methods for fisheries research. Our testing of this library found that the {\it optim} routine in R did not work on our 32-bit machines while it performed according to expectations on 64-bit systems. This glitch is beyond our control and we are awaiting a fix. Until then, we recommend you and your reviewers to run the examples from the SAFR library on 64-bit machines.\\ 


We hope that you will find this work interesting and relevant to our scientific community. \\

%  As a consequence, we estimated that the spawning stock biomass as well as recruitment spanned a large range of values. The emergence of a stronger competition in the prawn market created an interesting situation from the fishery research point of view in which we witnessed a reduction of fishing fleet by approximately 2/3 The article describes its application to a prawn fishery goes further than describing a new method  to estimate mortalitiescomprehensive analysis of all length measurements collected since 1989 during scientific surveys designed to assess the status of rock lobster in Torres Strait. It provides a new method that unable stock assessment scientists to (1) estimate Von Bertalanffy growth function parameters from length frequency data; (2) estimate uncertainties associated with those parameters and (3) determine which model best represents the variation of growth across the entire time series of data (currently, just over two decades). The results provide new insights on lobster somatic growth that are relevant to stock assessment and future work on the effect of climate change on this natural population. \\

\noindent Regards \\
%\noindent M. Kienzle%, J. McGilray and Y. Wang

\end{spacing}

%\hline
\vspace{3cm}

%\noindent Marco Kienzle\\
%Dept of Agriculture, Fisheries and Forestry of the Govt of Queensland\\ 
%Level A2, Ecosciences Precinct, Joe Baker St\\ 
%Dutton Park  QLD-4102\\
%Australia \\
%Telephone 07 3255 4227 Fax 3846 1207\\ 
%Email Marco.Kienzle@daff.qld.gov.au\\ 
%29 Kamarin st\\
%Manly-West QLD-4179\\
%Australia\\
%Email: Marco.Kienzle@gmail.com\\
%Phone: +61 4 05795090 \\ \\

\end{document}
