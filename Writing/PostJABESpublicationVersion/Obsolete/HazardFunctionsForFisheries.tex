%  This document must be used with LaTeX 2e!!!

%  See the documentation ``The biometrics class'' by
%  Josephine Collis for more information

%  Be sure to put the files biometrics.sty and biometrics.bst
%  in the same directory as your latex file.  Also put the .bib
%  file -- see below

\documentclass[12pt]{article}

\usepackage{biometrics2}
%\usepackage{natbib}
\usepackage{amsmath,epsfig,epsf,psfrag}
%\usepackage{a4wide,amsmath,epsfig,epsf,psfrag}
\usepackage[nolists]{endfloat}

%  put your commands here

\def\be{{\ensuremath\mathbf{e}}}
\def\bx{{\ensuremath\mathbf{x}}}
\def\bthet{{\ensuremath\boldsymbol{\theta}}}
\newcommand{\VS}{V\&S}
\newcommand{\tr}{\mbox{tr}}

\begin{document}

%  make sure that the document has 25 lines per page (it is 12 pt)

\setlength{\textheight}{575pt}
\setlength{\baselineskip}{23pt}

%  IMPORTANT -- place the \title, \author, etc, \begin{summary}...
%  \end{summary} \keywords statements in the order demonstrated below

\title{Hazard function models to estimate mortality rates affecting fish populations by maximum likelihood using age data,\\ with application to the sea mullet ({\it mugil cephalus}) fishery on the Queensland coast (Australia)}

%\author{}
%% \address{
%%  Department of Excavation,  University of Bedrock, 
%%  Bedrock, NV 99999, U.S.A.}
%% \email{flintstone@bedrock.edu}

%% \author{B. Rubble}
%% \address{
%% Department of Statistics, Bedrock State University \\
%%  Bedrock, NV 99999, U.S.A.}
%\email{brubble@bedrock.edu} 
% with more than 2 authors, give only one email address

%% \author{M. Slate}
%% \address{
%%  Department of Excavation,  University of Bedrock,
%%  Bedrock, NV 99999, U.S.A.}

%  you can put line breaks in the addresses or not as you choose

%\date{}


\maketitle

\begin{summary}
Fisheries management agencies around the world collect age data for the purpose of assessing the status of natural resources in their jurisdiction. Estimates of mortality rates are a key information to assess the sustainability of exploiting fish stocks. In medical research, failure rates are estimated using survival analysis, a statistical method seldom applied to stock assessment despite having the same goals as those of fisheries research. In this paper, we developed hazard functions to model the dynamic of a fishery. This method estimated all parameters necessary for stock assessment by maximum likelihood, including natural, fishing mortality rates and gear selectivity using age from a sample of fish caught by the fishery. This statistical approach was tested by Monte Carlo simulations to assert that it provided un-biased estimates of relevant quantities. An application to a sample of age from sea mullet ({\it Mugil cephalus}) caught between 2007 and 2014 in Queensland (Australia) provided the first estimate of natural mortality affecting this stock (M=0.22 $\pm$ 0.08 year$^{-1}$).
\end{summary}

\keywords{Monte Carlo; survival analysis; stock assessment}

\section{Introduction}

\cite{scimar42} and \fullcite{Pollock1989}

%\begin{acknowledgements}
%\end{acknowledgements}

%\begin{resume}
%The French version of the summary, if you want to translate it yourself.
%\end{resume}

%  This will create the references section.  To get this, you need
%  to use BiBTeX.  First, latex the document; if your latex file
%  is called example.tex, as this one is, e.g. latex example.
%  Create the file example.bib (we've already done this; you may
%  use this file as an example/template).  Then run BiBTeX on this file, e.g.
%  bibtex example in our case.  Then latex the document TWICE again, e.g. run
%  latex example TWICE.  Now look at the dvi file -- you will see that
%  the references in example.bib will have been included in a 
%  References section in the document in the Biometrics style.  The
%  file example.bbl will have been created.  The .bib file should
%  be located in the same directory as your latex file.

\bibliographystyle{/home/mkienzle/texmf/tex/biometrics} \bibliography{Biblio}

% appendices with and without titles and numbers

%% \startappendix
%% \appendix{Proof of Theorem 1}
%% This appendix will have a letter (number) and a title

%% \appendix*{}
%% This appendix has no title or letter (number)

\end{document}


