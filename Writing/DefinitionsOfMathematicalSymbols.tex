\label{Appendix:DefinitionsOfMathematicalSymbols}

This appendice contains definitions of some of the mathematical symbols used in previous sections

\begin{itemize}

\item $S_{i,j}$: a matrix of dimensions $n \times p$ ($i \in [1, n]$ and $j \in [1, p]$) containing a number of fishes that were aged and found to belong to specific age-groups $j$ in a particular year $i$. This matrix contains data belonging to $n+p-1$ cohorts, which by convention were labeled using $k$ varying from 1 on the top-right corner of the matrix to $n+p-1$ on the bottom-left (Tab.~\ref{Tab:Cohorts}). 

\begin{table}[ht]
\begin{center}

\begin{tabular}{c|cccccc}
%  \toprule
 \multicolumn{1}{c}{} & 1 & & \dots & & & $p$ \\
\cmidrule(r){2-7}
 1      & \dots   & \dots & \dots & 3     & 2       & 1      \\
        & \dots   & \dots & \dots & \dots & 3       & 2      \\
\vdots  & \dots   & \dots & \dots & \dots & 4       & 3      \\
        & \dots   & \dots & $k$   & \dots & \dots   & \dots  \\
        & \dots   & \dots & \dots & \dots & \dots   & \dots  \\
$n$     & $n+p-1$ & \dots & \dots & \dots & \dots   & \dots  \\
\end{tabular}

\caption{Convention used to associate each element of the catch at age matrix ($C_{i,j}$) with particular cohort referred to as with the number given in this table.}
\label{Tab:Cohorts}


\end{center}
\end{table}




The number of data in $S_{i,j}$ belonging to each cohort ($r_{k}$) varies from $1$ to ${\rm min}(n,p)$ and was determined as follow:

\begin{equation}
r_{k} = 
\begin{cases}
i - j + p \ {\rm if} \ k < {\rm min}(n,p)  \\ 
{\rm min}(n,p) \ {\rm if} \ {\rm min}(n,p) \leq k < {\rm max}(n,p) \\
j - i + n \ {\rm if} \ k \geq {\rm max}(n,p) \\
\end{cases}
\end{equation}


\noindent Each element of the $S_{i,j}$ matrix is uniquely identified using indices $i$ and $j$ ( $1 \leq i \leq n$ and $ 1 \leq j \leq p$) or indices $k$ and $l$ ( $ 1 \leq k \leq n+p-1 $ and $ 1 \leq l \leq r_{k}$ ), so for example
\begin{equation}
  \sum_{i,j} S_{i,j} = \sum_{k,l} S_{k,l}
\end{equation}

\item $P_{i,j}$: a matrix of dimensions $n \times p$ ($i \in [1, n]$ and $j \in [1, p]$) containing the proportion at age in the sample ($S_{i,j}$). Rows of this matrix sum to 1.

\begin{equation}
p_{i,j} = \frac{S_{i,j}}{\sum_{j} S_{i,j}}
\end{equation}

\item $F_{i,j}$ a matrix of fishing mortality with dimension $n \times p$ ($i \in [1, n]$ and $j \in [1, p]$). This matrix was constructed as the outer product of year specific fishing mortalities ($q \ E_{i}$) and selectivity at age ($s_{j}$):
\begin{equation}
F_{i,j} = q \ E_{i} \otimes s_{j}
\end{equation}

\end{itemize}
