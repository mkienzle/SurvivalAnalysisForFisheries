This section describes an application of survival analysis to matrices of catch at age, developed for the purpose of estimating catchability ($q$), selectivity at age ($s(t)$) and constant natural mortality ($M$). The matrix ($S_{i,j}$) containing a sample of fishes aged to belong to a particular age-group $j$ in year $i$ contains $n+p-1$ cohorts. These cohorts were indexed by convention using $k$ ($k \in [1, n+p-1]$) and an increasing number $r_{k}$ ($ 1 \leq r_{k} \leq {\rm min}(n,p)$)
identifying incrementally each age-group (see appendix p.~\pageref{Appendix:DefinitionsOfMathematicalSymbols} for more information). Each matrix $S_{i,j}$ has two cohorts with only 1 age-group representing them.

The derivation for a single cohort were the same as those presented in the previous section, here reproduced with indexations relative to a single cohort and accounting for selectivity

\begin{equation}
g_{k}(t; \theta) = \frac{q \ s(t) \ E(t) \times e^{-Mt-q\int_{0}^{t} s(t) \ E(t) \ dt}} {\sum_{l=1}^{r_{k}} \frac{q \ s_{k,l} \ E_{k,l}}{M+q \ s_{k,l} \ E_{k,l}} \bigl ( e^{-M \ a_{k,l}-q\int_{0}^{a_{k,l}}s(t) \ E(t) \ dt} - e^{-M \ a_{k,l}-q\int_{0}^{a_{k,l+1}}s(t) \ E(t) \ dt} \bigr )}
\end{equation}

{\bf This expression needs more justification.  $qs(t)E(t)$ should be $q s(t)E(t) /(M+q s(t)E(t))$?

Why not define}

$$ 
 P_l = \frac{q \ s_{k,l} \ E_{k,l}}{M+q \ s_{k,l} \ E_{k,l}} \bigl ( e^{-M \ a_{k,l}-q\int_{0}^{a_{k,l}}s(t) \ E(t) \ dt} - e^{-M \ a_{k,l}-q\int_{0}^{a_{k,l+1}}s(t) \ E(t) \ dt} \bigr )
$$
and $g_k = P_l / \sum_{l=1}^{r_{k}} P_l$.


{\bf The following likelihood $\mathcal{L}$  is in fact the conditional  likelihood as mentioned in the previous section.  To highligh the novelty,  it is better to put more details,
and it will be easier to read and appreciate by others.}



The likelihood function of a catch at age matrix was build using each pdf specific to each cohort ($g_{k}(t; \theta)$):

\begin{equation}
\mathcal{L} = \prod_{k=1}^{n+p-1} \prod_{l=1}^{r_{k}}  \bigl ( \int_{t=a_{k,l}}^{t=a_{k,l+1}} g_{k}(t; \theta) \ dt \bigr ) ^ {S_{k,l}}
\end{equation}

The expression above is equivalent to
\begin{equation}
\mathcal{L} = \prod_{i,j} P_{i,j} ^ {S_{i,j}}
\end{equation}

\noindent where the $P_{i,j}$ are the probabilities of observing a fish of a given age $j$ in year $i$ given by the hazard model. In this likelihood, the $P_{i,j}$ sum to 1 along the cohort instead of summing to 1 for each year as described for the multinomial likelihood in \cite{Four82a}. \\

This method was implemented in R \citep{R} in the package Survival Analysis for Fisheries Research (SAFR). Numerical application of this method are available using the following commands: {\bf library(SAFR); example(llfunc3); example(llfunc4); example(llfunc5);}.\\
