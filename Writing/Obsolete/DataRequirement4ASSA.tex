Methods for stock assessments have been developed for a variety of data to estimate fishing mortality and the status of a fish population \citep{DowP2007}. Age-structured methods often require the number of years of age data collection to exceed the number of age-group existing in the population to allow for one or more cohorts to have gone entirely through the fishery (Tab.~\ref{tab:SAmethodRequirement}).\\

Doubleday (1976) established that catch-at-age data alone are insufficient to estimate stock biomass reliably because biomass and fishing mortality are negatively correlated \citep{Maunder201361}. As a consequence, most stock assessments using catch at age data also require an index of abundance. Ideally, this index should be independent from commercial catch and effort data. But in many cases where such fishery independent data are not available, researchers have used catch per unit of effort or standardised catch per unit of effort in place of the index of abundance (CITE).\\ 
