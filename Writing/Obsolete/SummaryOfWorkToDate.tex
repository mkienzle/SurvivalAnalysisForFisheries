Outcomes of the preliminary investigations (20 days)

\begin{itemize}
\item age structure stock assessment model applies to catch at age data spanning a number of years larger than the number of age-groups (see \cite{CASAL12m, mill99m, Four82a} for examples).
\item LTMP catch at age data for sand whiting covers 7 years (2007--2013) and 13 age-groups (0--12): at least six more years of data collection are required until stock assessment methods found in the literature can be applied to them.\footnote{This sounds contractory to \cite{HoyR2000} at p.8 which stated in 2000 that "The state's commercial fisheries catch and effort database does not yet span a sufficiently long time period to be of much value in assessments involving biomass dynamic modelling. Stock assessments procedures therefore need to incorporate catch-at-age and size-at-age techniques so that changes in population structure can be tracked from year to year."}
\item no framework was readily available to process these data in less than 20 days: substantially more time has to be allocated to this research topic to achieve the goals of LTMP.
\item Some of the same problem identified more than 10 years ago by \cite{HoyR2000} still persist, namely "(Recreational) estimates are biennnial (or worse) rather than annual, so are difficult to use effectively in stock assessment models."
\end{itemize}

\noindent Possible extension of the work

\begin{itemize}
\item a stock assessment that doesn't use age-data (such as the delay difference model) could be applied to catch and effort. 
\item a new method to use estimate mortalities from catch at age matrices (with $n<p$) could be develop
\item the two previous items could be integrated into a single (hybrid) method to estimate the dynamic of sand whiting. The time necessary to develop such non-standard method might well be in excess of 2 years of work.
\end{itemize}
