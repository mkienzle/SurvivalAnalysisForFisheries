Sand whiting  ({\it Sillago ciliata}, Cuvier 1829) is primarily estuarine and near shore species harvested by both commercial and recreational fishing sectors\citep{JAI:JAI12478}. In Queensland, the majority of the commercial catch come from bays: Hervey Bay accounted for 47\% of total catch since 1988 and Moreton Bay accounted for 37\% (Fig.~\ref{fig:CatchPropertionPerRegion}). Small amount of this species are caught in between these two areas (Fig.~\ref{fig:TotalCatch-byYear-byRegion}). Catch-per-unit-effort (CPUE) were systematically larger in Hervey Bay than Moreton Bay (Fig.~\ref{fig:VariationOfCPUE-byYear-byRegion}). These two facts together suggested that Hervey Bay and Moreton Bay should be treated as separate entities for stock assessment purposes. This idea was further supported by a professional fishermen with over 40 years experience fishing Moreton Bay reporting that sand whiting from Hervey Bay differ from Moreton Bay, the former being a smaller variety known as "maget" (J. Page pers. comm.; do this correspond to {\it S. ciliata} and {S. analis} ?). Available data indicated that sand whiting were mostly caught using gillnets until the late 1990s but later ring netting became dominant in Hervey Bay. While tunnel netting became the prefered fishing method for sand whiting in Moreton Bay (Fig.~\ref{fig:TotalCatch-byYear-HerveyAndMoretonBay}). Total catches have been fairly constant since 1988 with a mean catch around 130 $\pm$ 33 tonnes in Hervey Bay and 100 $\pm$ 27 tonnes in Moreton Bay. Catches have been slightly declining in recent years in both regions. \\

Commercial catch and CPUE showed a strong seasonal variation peaking in winter. Sand whiting is known by fishermen as a cold temperature specie that becomes present in Moreton Bay during the winter months. Sand whiting starts aggregating in or around May, probably for reproductive purposes, until about September after which it disappears from the bay, possibly going back up the creeks where water keeps cooler during the summer months (J. Page pers. comm.). These fish are fertile throughout the fishing season (May--September) and are spawning towards the end.\\



