Fish can be assigned an age by examining its otolith, which is found just below its brain. Fish otoliths deposit calcium carbonate through time, thus increasing in size each year of their life. Microscopic observation of otolith sections often reveal alternate opaque and translucent zones, which can be used to assign individual fish to a particular age group. \\ 

Sampling programs in fisheries research centers around the world aim to collect a representative sample of fish each year to determine the distribution of age of any species of interest. In most cases, the data are binned into age-groups of width 1 year. For this reason, we split the lifespan of cohorts from their birth ($t \in [0;\infty]$) into $n$ yearly intervals from $a_{1}=0$ to the maximum age of $a_{n+1}$ years. While the theory presented in this document used that particular subdivision of time ($t$), unequal ones also applies. In fact, an un-equal subdivision of time was used for the sea mullet case study.
 
