Fish aging is possible thanks to a little bone called the otolith which is present in their ears. An otolith accumulate materials and increases in size throughout the entire lifespan of a fish. Microscopic analyses of sections of an otolith shows a series of marks, similar to tree rings, that can be used to assign each individual to a specific age-group.\\

Most fisheries institute around the world have a sampling program dedicated to collect a representative sample of fish each year to determine the distribution of age of any species of interest. In most cases, the data are binned into age-groups of width 1 year. For this reason, we split the lifespan of cohorts from their birth ($t \in [0;\infty]$) into $n$ yearly intervals from $a_{1}=0$ to the maximum age of $a_{n+1}$ years. While the theory presented in this document used that particular subdivision of time ($t$), un-equal ones also applies. In fact, an un-equal subdivision of time was used for the sea mullet case study.
 
