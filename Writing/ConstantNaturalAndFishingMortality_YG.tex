The exponential decrease in abundance of individuals belonging to a single cohort due to constant natural ($M$) and fishing ($F$) mortalities was described from a survival analysis point of view \citep{cox84b} using a constant hazard function of time ($t$) and parameters $\theta$

\begin{equation}
h(t; \theta) = M + F
\end{equation}

The probability density function (pdf) describing survival from natural and fishing mortality is

\begin{equation}
f(t; \theta) = (M + F) \ e^{-(M+F)t} = \underbrace{M \times e^{-(M+F)t}}_{=f_{1}(t; \theta)} + \underbrace{F \times e^{-(M+F)t}}_{=f_{2}(t; \theta)}
\end{equation}

Since age data belonging to individuals dying from natural causes are generally not available to fisheries scientists, we used only the component of the pdf that relates to fishing mortality ($f_{2}(t; \theta)$). This component of $f(t; \theta)$ integrates over the entire range of $t$ to
\begin{equation}
\int_{t=0}^{t=\infty} f_{2}(t; \theta) \ dt  = \frac{F}{M+F}.
\end{equation}

Hence, the probability density of being caught at age $t$ is  (by normalizing $f_{2}(t; \theta)$),
\begin{eqnarray}
g(t; \theta) &=& \frac{M+F}{F} \ f_{2}(t; \theta) \\
             &=& \frac{M+F}{F} \ F \times e^{-(M+F)t} \\
             &=& f(t; \theta),
\end{eqnarray}
which is the same as the original pdf.

The probability of being caught during age $(a_i, a_{i+1})$ is
$$
P_i =   \int_{t=a_{i}}^{t=a_{i+1}} f(t; \theta) \ dt
   = \exp\{M+F)a_i)\} H_{a_i} (a_{i+1} -a_i).
$$
where $H_t(s)$ is the conditional probability of surviving up to age $A$ conditional on being alive at age $a$ ($A > a$), specifically, $H_a(A) =  1-\exp(A-a)$.



%Following Fisher's definition \citep{edwards1992likelihood},

Suppose a total of $S$ fish being caught with $S_{i}$ being the number of fish at age between $a_i$ and $a_{i+1}$. The overall marginal likelihood of this sample
%% \begin{equation}
%% \mathcal{L} = \prod_{i=1}^{n} \bigl ( \int_{t=a_{i}}^{t=a_{i+1}} g(t; \theta) \bigr ) ^ {a_{i}}
%% \end{equation}

\begin{eqnarray}
\mathcal{L}  &=& \prod_{i=1}^{n} \bigl ( \int_{t=a_{i}}^{t=a_{i+1}} f(t; \theta) \ dt \bigr ) ^ {S_{i}} \\
            &=& \prod_{i=1}^{n} P_{i} ^ {S_{i}}
\end{eqnarray}

This is often referred to as the likelihood of a multinomial probability ($P_{i}$) where $P_{i}=\int_{t=a_{i}}^{t=a_{i+1}} f(t; \theta) \ dt$.

{\bf  This is not multinominal,  which requires $\sum P_i = 1$

Because the fish younger than $a_1$ is not considered are not taken into account (they exist theoretically in the population). The likelihood is not complete and hence incorrect!
}

The way of getting around this is to look at the relative distribution between these age groups. The younger fish may live somewhere else.

Let $p_i=P_i/\sum_{j=1}^n P_j$, the relative proportions among those age groups of interest.
Conditional on the total catch $S$ with age between age $a_1$ and $a_n$, the age frequency of the total sample $S$ follows the following multinominal distribution (up to a constant $S!/s_1!...S_n!$) (see Wang, 1999).
$$
 \prod_{i=1}^{n} p_{i} ^ {S_{i}}.
$$


The logarithm of the likelihood was
%% \begin{eqnarray}
%% {\rm log}(\mathcal{L}) &=& \sum_{i=1}^{n} a_{i} \ {\rm log} \bigl ( \int_{t=a_{i}}^{t=a_{i+1}} g(t; \theta) \bigr ) \\
%%                        &=& \sum_{i=1}^{n} a_{i} \ {\rm log} \bigl ( \int_{t=a_{i}}^{t=a_{i+1}} \frac{f_{2}(t; \theta)}{1 - \frac{M}{M+F}} \bigr ) \\
%%                        &=& \frac{1}{1 - \frac{M}{M+F}} \sum_{i=1}^{n} a_{i} \ {\rm log} \bigl ( \int_{t=a_{i}}^{t=a_{i+1}} f_{2}(t; \theta) \bigr ) \\
%%                        &=& \frac{1}{1 - \frac{M}{M+F}} \sum_{i=1}^{n} a_{i} \ {\rm log} \bigl ( \frac{F}{M+F} (e^{-(M+F) \times a_{i}} - e^{-(M+F) \times a_{i+1}}) \bigr ) \\
%% \end{eqnarray}

\begin{eqnarray}
{\rm log}(\mathcal{L}) &=& \sum_{i=1}^{n} S_{i} \ {\rm log} \bigl ( \int_{t=a_{i}}^{t=a_{i+1}} f(t; \theta) \ dt \bigr ) \\
                       &=& \sum_{i=1}^{n} S_{i} \ {\rm log} \bigl ( \int_{t=a_{i}}^{t=a_{i+1}} (M + F) \ e^{-(M+F)t} \ dt \bigr ) \\
                       &=& \sum_{i=1}^{n} S_{i} \ {\rm log} \bigl ( e^{-(M+F) \times a_{i}} - e^{-(M+F) \times a_{i+1}} \bigr ) \\
%                       &=& \sum_{i=1}^{n} S_{i} \ {\rm log} \bigl ( e^{-(M+F) \times a_{i}} (1 - e^{-(M+F)}) \bigr ) \\ % this is not correct because $a_{i+1} \neq a_{i}+1$
\end{eqnarray}

This development illustrated an application of survival analysis to estimate mortality rates affecting a cohort of fish by maximum-likelihood using a sample of catch at age. This method was implemented in R \citep{R} in the package Survival Analysis for Fisheries Research (SAFR) provided as supplement material. Numerical application were made available using the following commands: {\bf library(SAFR); example(llfunc1)}.\\

Natural and fishing mortality cannot be disentangled with catch data only but the next section will show that the provision of {\bf varying}  effort data allowed to estimate both catchability ($q$) and natural mortality.
