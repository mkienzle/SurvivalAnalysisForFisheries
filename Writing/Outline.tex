\documentclass[11pt]{article}
\usepackage{verbatim}
%\usepackage{html}
%\usepackage{wasysym}
\usepackage{natbib}
\usepackage{graphicx}
\usepackage{ifpdf}
\ifpdf
\DeclareGraphicsExtensions{.pdf,.png,.jpg}
\else
\DeclareGraphicsExtensions{.eps}
\fi

\usepackage{setspace}
\pagestyle{myheadings}
\markright{Manuscript outline}

%\setlength{\oddsidemargin}{.1375in}
%\setlength{\evensidemargin}{.1375in}
\setlength{\oddsidemargin}{.15cm}
\setlength{\evensidemargin}{.15cm}
\setlength{\textwidth}{16cm}
\setlength{\textheight}{24cm}

\setlength{\headsep}{0.9cm}
\setlength{\jot}{0.4cm}
\setlength{\topsep}{0.6cm}

\flushbottom

\long\def\symbolfootnote[#1]#2{\begingroup%
\def\thefootnote{\fnsymbol{footnote}}\footnote[#1]{#2}\endgroup} 

\long\def\symbolfootnotemark[#1]{\begingroup%
\def\thefootnote{\fnsymbol{footnote}}\footnotemark[#1]\endgroup} 

\thispagestyle{empty}% Remove page numbering

\begin{document}

\title{Outline}
\maketitle

%%%%%%%%%%%%%%%%%%%%%%%%%%%%%%%%%%%%%%%%%%%%%%%%%%%%%%%%%%%%%%%%%%%%%%%%%%%%%%%%%%%%%%%%%%%%%%%%%%%%%%%%%%%%%%%%%%%%%%%%%%%%%%%%%%%%%%%%%%%%%%%%%%%%%%%%%%

\section{Why are we doing this work ? why is this topic important ? Think about the key question you and your audience might have}

\begin{itemize}
\item The purpose of stock assessment is to estimate mortality rates
\item The central model is an exponential decay (Baranov model)
\item Likelihood approach have been widely used to estimate parameter. It is central to the integrated approach (CITE)
\item Likelihood approach requires one to look at problems from a statistical perspective
\item Given that statistic were embraced for the likelihood, it is interesting to see that the statistical counterpart of the exponential decay, in the name of the exponential pdf, was never used. 
\item The branch of statistics specializing in estimating mortality rates is known as survival analysis.
\item Survival analysis is widely applied to medical research

\item This paper describes an application of survival analysis to the analyse of catch at age
\end{itemize}

%%%%%%%%%%%%%%%%%%%%%%%%%%%%%%%%%%%%%%%%%%%%%%%%%%%%%%%%%%%%%%%%%%%%%%%%%%%%%%%%%%%%%%%%%%%%%%%%%%%%%%%%%%%%%%%%%%%%%%%%%%%%%%%%%%%%%%%%%%%%%%%%%%%%%%%%%%
\section{Results}

\begin{itemize}
\item Likelihood of catch at age
\item Ability to estimate parameter from data matrices with $n < p$ 
\end{itemize}

%%%%%%%%%%%%%%%%%%%%%%%%%%%%%%%%%%%%%%%%%%%%%%%%%%%%%%%%%%%%%%%%%%%%%%%%%%%%%%%%%%%%%%%%%%%%%%%%%%%%%%%%%%%%%%%%%%%%%%%%%%%%%%%%%%%%%%%%%%%%%%%%%%%%%%%%%%
\clearpage
\newpage
\section{Detailed content of the article}


\noindent{\bf 1. Introduction}
\begin{itemize}

\item (CITE) used a statistical approach to catch at age analysis assuming these data were distributed according to a multinomial distribution. While this model recognise the presence of multiple age-groups in the data, it fails to account for properties associated with the Baranov equation such as for example that the proportion of individual surviving through time is a declining function of time.
\item Multinomial model
\item Baranov model

\end{itemize}

\noindent {\bf 2. Materials and methods}\\

\noindent {\it 2.1 Likelihood method}
\begin{itemize}
\item very simple case on a single cohort
\item estimating M, q and selectivity on a single cohort
\item Multiple cohorts and the separability hypothesis
\item Estimator of recruitment
\end{itemize}

\noindent {\it 2.2 Simulations testing}\\
\begin{itemize}
\item The datasets
\item Data uncertainty
\end{itemize}

\noindent {\it 2.3 Application to a case-study}\\
\begin{itemize}
\item
\end{itemize}

\noindent {\bf 3. Results}\\
\noindent {\it 3.1 Simulation testing}
\begin{itemize}
\item
\end{itemize}

\noindent {\it 3.2 Case-study}
\begin{itemize}
\item
\end{itemize}

\noindent {\bf 4. Discussion}\\

\begin{itemize}
\item
\end{itemize}


%%%%%%%%%%%%%%%%%%%%%%%%%%%%%%%%%%%%%%%%%%%%%%%%%%%%%%%%%%%%%%%%%%%%%%%%%%%%%%%%%%%%%%%%%%%%%%%%%%%%%%%%%%%%%%%%%%%%%%%%%%%%%%%%%%%%%%%%%%%%%%%%%%%%%%%%%%%%
\section{Title}
Yet another general theory for the analysis of catch at age

%%%%%%%%%%%%%%%%%%%%%%%%%%%%%%%%%%%%%%%%%%%%%%%%%%%%%%%%%%%%%%%%%%%%%%%%%%%%%%%%%%%%%%%%%%%%%%%%%%%%%%%%%%%%%%%%%%%%%%%%%%%%%%%%%%%%%%%%%%%%%%%%%%%%%%%%%%%%

\noindent {\bf 5. Citations}\\
\begin{itemize}
\item On the other hand, if it is the parameters of a selectivity function that are estimated (rather than the individual selectivities), it is possible to estimate the natural mortality rate internally with the model studied here even when fishing mortality is constant, and with deterministic data the estimate is of course correct if the function is correctly specified.... The resulting estimate of the natural mortality rate is clearly artificial because it relies entirely on the precise shape of the specified selectivity function rather than on any information in the data. This side effect of using a parametric selectivity function ({\it i.e.} making the natural mortality reate appear to be estimable when in fact it is not) is another reason to prefer individual selectivities. In discussion, Estimating natural mortality internally, p. 1730 - \citep{doi:10.1139}
\end{itemize}
%% Bibliography
\bibliographystyle{plainnat}
\bibliography{/home/mkienzle/mystuff/Bibliography/long,/home/mkienzle/mystuff/Bibliography/Biblio}

\end{document}
