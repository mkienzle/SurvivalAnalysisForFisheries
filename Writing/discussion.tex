This application of survival analysis to fisheries research provided a novel approach to develop maximum likelihood estimators of natural, fishing mortality rates and gear selectivity from age data. Monte Carlo simulations showed that it provided unbiased estimates of natural mortality and catchability over a wide range of simulated values. \\

% Comparison of neg likelihood
The comparison between the negative log-likelihood from the survival analysis approach with the multinomial likelihood \citep{Four82a} suggested that the former offered a better model to represent the data. This comparison was made using the best possible outcome for the multinomial likelihood because it used the simulated proportions of individuals at age in place of the probabilities to compute the likelihood. Arguably, this approach gave a substantial advantage to the multinomial likelihood over the survival analysis: no one would reasonably expect any estimation method to systematically provide exactly the proportion at age in the catch using a sample of the data. Therefore, the present comparison really focused on which probabilities to use in the likelihood function, whether they should sum to 1 in each years along age-groups or along cohorts. Despite the strong advantage given to the multinomial likelihood, the results suggested that simulated data according to Baranov's catch equation were fundamentally better fitted by a statistical method that modelled the exponential decay of individuals along cohort rather than by one that assumed the data followed a multinomial probability distribution specific to each year.\\

Weighting of sample provided unbiased estimates of natural mortality and catchability. Mortality estimates, in particular fishing mortality, depended on the magnitude of catch. The unrealistic sampling strategy which assumed that all catch data would be in front of the experimenter at once for sampling, accounted automatically for variation of catch and effort in each year because the abundance of each age-group in the catch determined the probability to choose at random an individual belonging to any age-group. In practical application of survival analysis to fishery research, weighting is necessary because one cannot know {\it a priori} the magnitude of catch in coming years. \\

%%%%%%%%%%%%%%%%%%%%%%%%%%%%%%%%%%%%%%%%%%%%%%%%%%%%%%%%%%%%%%%%%%%%%%%%%%%%%%%%%%%%%%%%%%%%%%%%%%%%%%%%%%%%%%%%%%%%%%%%%%%%%%%%%%%%%%%%%%%
%%% Simulations
%%%%%%%%%%%%%%%%%%%%%%%%%%%%%%%%%%%%%%%%%%%%%%%%%%%%%%%%%%%%%%%%%%%%%%%%%%%%%%%%%%%%%%%%%%%%%%%%%%%%%%%%%%%%%%%%%%%%%%%%%%%%%%%%%%%%%%%%%%%

The simulations used a logistic gear-selectivity to generate and fit the data although we would have preferred to generate data from a wide range of possible gear-selectivity functions or even using non-parametric procedures. Simulations showed that gear selectivity were the most difficult parameters to estimate. The sea mullet case study was in fact not fitted with a logistic curve but selectivity were estimated through a tedious process to search each proportion retained at age that best fitted the data as measured by the likelihood. This process could not be automatized into the simulation testing framework to provide automatic identification of gear-selectivity. This aspect of the analysis was left out of the present manuscript for future work. Criticisms that this somewhat simplified the problem would be correct. But the current article was designed as an introduction to the application of survival analysis to fisheries research not one that solves all problems at once. As such, the likelihood approach presented in this manuscript provides a method to identify the gear selectivity that best fit the data, just not an automatic one. \\ 

The estimations of natural mortality and catchability using data from a fixed number of fish every year were biased probably because data were simulated with large variations of recruitment and fishing effort, resulting in large variation of catches between years. Hence weighting number at age samples by total catch probably introduced large correction to the datasets in many simulations. The effect of weighting on parameter estimates was noticeable also in the case of the mullet fishery where the coefficient of variation of catch was 12.2\%.\\



%%%%%%%%%%%%%%%%%%%%%%%%%%%%%%%%%%%%%%%%%%%%%%%%%%%%%%%%%%%%%%%%%%%%%%%%%%%%%%%%%%%%%%%%%%%%%%%%%%%%%%%%%%%%%%%%%%%%%%%%%%%%%%%%%%%%%%%%%%%
%%% Sea Mullet
%%%%%%%%%%%%%%%%%%%%%%%%%%%%%%%%%%%%%%%%%%%%%%%%%%%%%%%%%%%%%%%%%%%%%%%%%%%%%%%%%%%%%%%%%%%%%%%%%%%%%%%%%%%%%%%%%%%%%%%%%%%%%%%%%%%%%%%%%%%

% about Sea Mullet stock structure
It was surprising that this analysis of Sea Mullet data from the QLD fishery estimated similar values, in particular gear selectivity estimates, to the most recent stock assessment performed on a much larger and diverse dataset that included data from New South Wales (NSW) \citep{Bell2005r}. \cite{Lester2009129} suggested, using parasites, that the bulk of Sea Mullet caught in Queensland fishery is based on local fish populations and not migrating from NSW. While genetic analyses could not identify differences in single nucleotide polymorphism between samples from south QLD and NSW \citep{kruck2013a}. A clarification of the boundaries of stock of Sea Mullet on the Australian east coast should precede further data analysis and development of management strategies for this fishery.\\

% estimate of natural mortality
The sensitivity analysis to data truncation showed a weak trend in increasing uncertainty associated with natural mortality. Intuition would have suggested that old, rare, individuals provided valuable information about mortality hence increasing our knowledge on survival. The results of data truncation suggested the opposite, that inclusion of older age-groups containing few or no observations increased our uncertainties on mortality estimates. Possibly this lack of data induced large uncertainties on gear selectivity for older age group because lack of observations in those could be the result of high mortality or low selectivity. Uncertainties in that aspect of the model might have propagated into other components, increasing uncertainty about natural mortality. \\

%%%%%%%
% Future work
%%%%%%%


This likelihood method might find its place naturally into integrated stock assessment \citep{Maunder201361} as it provided an efficient method to deal with samples of age data. Applications of survival analysis to fishery data could be expanded further, a particular area of interest for future development would be to use this method to derive recruitment estimates based on the probabilities estimated from survival analysis and total catch data from the fishery to generate the most likely time series of recruitment.
