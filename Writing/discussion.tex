This application of survival analysis to fisheries research provided an effective approach to develop maximum likelihood estimators of natural and fishing mortality rates, and gear selectivity, from age data. Monte Carlo simulations showed that it provided unbiased estimates of natural mortality and catchability over a wide range of simulated values. \\

% Comparison of neg likelihood
The comparison between the negative log-likelihood from the survival analysis approach with the multinomial likelihood \citep{Four82a} suggested that the former offered a better model to represent the data. This comparison was made using the best possible outcome for the multinomial likelihood because it used the simulated proportions of individuals at age in place of the probabilities to compute the likelihood. Arguably, a substantial advantage was given to the multinomial likelihood over the survival analysis in this comparison because no one would reasonably expect any estimation method to systematically provide exactly the proportion at age in the catch using a sample of the data. Therefore, the present comparison really focused on which probabilities to use in the likelihood function, whether they should sum to 1 in each year along age-groups or along cohorts. Despite the strong advantage given to the multinomial likelihood, the results suggested that simulated data according to Baranov's catch equation were fundamentally better fitted by a statistical method that modelled the exponential decay of individuals along cohorts rather than by one that assumed the data followed a multinomial probability distribution specific to each year.\\

Weighting of the sample provided unbiased estimates of natural mortality and catchability. Mortality estimates, in particular fishing mortality, depended on the magnitude of catch. The unrealistic sampling strategy which assumed that all catch data would be in front of the experimenter at once for sampling, accounted automatically for variation of catch and effort in each year because the abundance of each age-group in the catch determined the probability to choose at random an individual belonging to any age-group. In practical application of survival analysis to fishery research, weighting is necessary because one cannot know {\it a priori} the magnitude of catch in coming years. \\

%%%%%%%%%%%%%%%%%%%%%%%%%%%%%%%%%%%%%%%%%%%%%%%%%%%%%%%%%%%%%%%%%%%%%%%%%%%%%%%%%%%%%%%%%%%%%%%%%%%%%%%%%%%%%%%%%%%%%%%%%%%%%%%%%%%%%%%%%%%
%%% Simulations
%%%%%%%%%%%%%%%%%%%%%%%%%%%%%%%%%%%%%%%%%%%%%%%%%%%%%%%%%%%%%%%%%%%%%%%%%%%%%%%%%%%%%%%%%%%%%%%%%%%%%%%%%%%%%%%%%%%%%%%%%%%%%%%%%%%%%%%%%%%

The Monte Carlo simulations used a logistic gear-selectivity to generate and fit the data although we would have preferred to generate data from a wide range of possible gear-selectivity functions or even using non-parametric procedures. Simulations showed that gear selectivity were the most difficult parameters to estimate. The sea mullet case study was in fact not fitted with a logistic curve but selectivity were estimated through a tedious process to search each proportion retained at age that best fitted the data as measured by the likelihood. This process could not be automatized into the simulation testing framework to provide automatic identification of gear-selectivity. This aspect of the analysis was left out of the present manuscript for future work. Criticisms that this somewhat simplified the problem would be justified. But the current article was designed as an introduction to the application of survival analysis to fisheries research, not one that solves all problems at once. As such, the likelihood approach presented in this manuscript provides a method to identify the gear selectivity that best fit the data, just not an automatic one. \\ 

%The estimations of natural mortality and catchability using data from a fixed number of fish every year were biased probably because data were simulated with large variations of recruitment and fishing effort, resulting in large variation of catches between years. Hence weighting number at age samples by total catch probably introduced a large and useful correction to the datasets in many simulations. The effect of weighting on parameter estimates was noticeable also in the case of the mullet fishery where the coefficient of variation of catch was 12.2\%.\\



%%%%%%%%%%%%%%%%%%%%%%%%%%%%%%%%%%%%%%%%%%%%%%%%%%%%%%%%%%%%%%%%%%%%%%%%%%%%%%%%%%%%%%%%%%%%%%%%%%%%%%%%%%%%%%%%%%%%%%%%%%%%%%%%%%%%%%%%%%%
%%% Sea Mullet
%%%%%%%%%%%%%%%%%%%%%%%%%%%%%%%%%%%%%%%%%%%%%%%%%%%%%%%%%%%%%%%%%%%%%%%%%%%%%%%%%%%%%%%%%%%%%%%%%%%%%%%%%%%%%%%%%%%%%%%%%%%%%%%%%%%%%%%%%%%

% estimate of mortality rates
The model best supported by the mullet dataset estimated natural mortality equal to $0.22 \pm 0.08$. This is the first estimate of natural mortality for mullet in Australia. Previous to this estimation, it was customary to use the natural mortality estimate produced by \cite{Hwang82a} for the mullet fishery in Taiwan using linear regression (M=0.33 year$^{-1}$) which fall within 2 S.D. of the estimate for the Queensland fishery. The model that fitted best the mullet data estimated catchability in the ocean to be 16 times larger than in estuaries. This is consistent with fishermen reporting very large catches from their ocean beach operations, up to 40 tonnes per haul.\\

%% % about Sea Mullet stock structure
%% This analysis of sea mullet data from Queensland fishery estimated similar values of gear selectivity to the most recent stock assessment \citep{Bell2005r} performed on a much larger and diverse dataset that included data from New South Wales (NSW). On the other hand, genetic analyses could not identify differences in single nucleotide polymorphism between samples from south QLD and NSW \citep{kruck2013a}. A clarification of the boundaries of stock of sea mullet on the Australian east coast should precede further data analysis and development of management strategies for this fishery.\\


%The sensitivity analysis to data truncation showed a weak trend in increasing uncertainty associated with natural mortality. Intuition would have suggested that old, rare, individuals provided valuable information about mortality hence increasing our knowledge on survival. The results of data truncation suggested the opposite, that inclusion of older age-groups containing few or no observations increased our uncertainties on mortality estimates. Possibly this lack of data induced large uncertainties on gear selectivity for older age group because the lack of observations in those could be the result of high mortality or low selectivity. Uncertainties in that aspect of the model might have propagated into other components, increasing uncertainty about natural mortality. \\

%%%%%%%
% Future work
%%%%%%%


This likelihood method may well find its place naturally into integrated stock assessment \citep{Maunder201361} as it provided an efficient method to deal with samples of age data. Applications of survival analysis to fishery data could be expanded further. A particular area of interest would be to derive recruitment estimates using the probabilities estimated by survival analysis and total catch from the fishery.
