Weighting the numbers of sampled fish each year by total catch (sampling strategy 2 - weighted sample) performed as well as the benchmark sampling strategy 1 (Fig.~\ref{fig:Estimating-NaturalMortality} and Fig.~\ref{fig:Estimating-Catchability}). By contrast, estimations using a fixed number of fish each year were biased suggesting that weighting by catch is necessary in practical applications of the survival analysis approach. \\

This weighting of age-data samples considerably reduced the uncertainty on natural mortality estimates (Fig~\ref{fig:Estimating-NaturalMortality}). Increasing the number of samples reduced uncertainty associated with natural mortality estimates. \\ 

Estimates of catchability were much more consistent across the range of values tested (1--10 $ \times 10^{-4}$) for all methods (Fig.~\ref{fig:Estimating-Catchability}). The bias of the unweighted approach was often similar to that of the weighted one. But the uncertainty associated with the former approach was much larger than the latter. For both strategy 1 and strategy 2 with weighting, the benefit of increasing sampling size were very noticeable up to a 1000 fish aged but less so beyond that.\\

The comparison between the likelihood function from survival analysis and the multinomial likelihood (Fig.~\ref{fig:ComparisonOfNegLL}) showed that, apart sampling strategy 2 which provided biased estimates, the approach using survival analysis provided in the majority of cases smaller negative log-likelihood values than the multinomial likelihood. The substantial advantage given the multinomial likelihood in this comparison played an important role at low sampling intensity where the assumption that proportion at age was known perfectly artificially improved its performance in most difficult situations. This artificial advantage faded away as the simulated sample sizes were increased resulting in the survival analysis approach outperforming the multinomial likelihood. \\

