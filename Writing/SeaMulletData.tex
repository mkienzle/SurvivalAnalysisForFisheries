The straddling sea mullet ({\it Mugil cephalus}) population stretches along the east coast of Australia, with most landings occurring between 19$^{o}$S (approx. Townsville) and 37$^{o}$S (roughly the border between New South Wales and Victoria). Following recommendations from \cite{Bell2005r}, the scientific survey design was modified from 2007 to include both estuarine and ocean habitats in order to provide representative demographic statistics for Queensland component of this fishery. Number at age obtained by otolithometry (Tab.~\ref{tab:Mullet-NbAtAge}) were analyzed to estimate natural mortality, catchability and gear selectivity. \\

Sea mullet spawn in oceanic waters adjacent to ocean beaches from May to August each year. By convention, the birth date was assumed to be on July 1$^{st}$ each year. Otolith ring formation occurs during spring and early summer (September to December). Biologists have come to the conclusion that the first identifiable ring is formed 14 to 18 months after birth, all subsequent rings forming at a yearly schedule. So each fish in the sample was assigned to an age group based on ring counts and translucent material at the margin of otoliths. Age group 0--1 comprised fish up to 18 months old ($a_{1}=18$ months) while all subsequent age groups spanned 12 months ($a_{2} = 30$ months, $a_{3}= 42$ months, etc ...).\\

Sensitivity of survival analysis estimates to these data, a matrix containing 7 years and 16 age-groups, were performed by truncating the dataset in 2 ways to assess the robustness of the method to varying number of years and age-groups. The first truncation removed the last and last-two years of data to evaluate the sensitivity of parameters estimates to addition/omission of data in order to anticipate possible effects of future addition of newly available data. The second truncation removed older age-groups from 10--11 to 15--16 to evaluate the importance of few old fish on natural mortality estimates as one could think {\it a priori} that these longer-lived individuals provided a lot of information on mortality.\\


