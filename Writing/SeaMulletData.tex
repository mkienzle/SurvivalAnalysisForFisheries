The straddling sea mullet ({\it Mugil cephalus}) is caught along the east coast of Australia, with most landings occurring between 19$^{o}$S (approx. Townsville) and 37$^{o}$S (roughly the border between New South Wales and Victoria). The most noticeable feature of the biology of this species is a massive northward spawning migration of the stocks along the coast during autumn \citep{Kesteven53a}. Tagging experiments revealed that 90\% of tagged animals travelled less than 85 km during the migration season \citep{Kesteven53a}. Analyses of parasites concluded that the bulk of sea mullet caught in Queensland fishery is based on local fish populations and not migrating from New Soth Wales \citep{Lester2009129}. Following recommendations from \cite{Bell2005r}, an existing (1999--2004) scientific survey design was modified from 2007 onward to include both estuaries and ocean habitats in order to provide representative demographic statistics for the Queensland component of this fishery. Samples were collected in both habitats (Tab.~\ref{tab:Mullet-NbAtAge}). Age varied between 0 and 16 years. A 14+ age-group was created to combine the small number of observations in the older age-groups. These data were weighted by catch in each year and habitat to obtain a dataset representative of the entire catch in this fishery. \\

Sea mullet are thought to spawn in oceanic waters adjacent to ocean beaches from May to August each year. By convention, the birth date was assumed to be on July 1$^{st}$ each year. Opaque zones are thought to be deposited on the otolith margin during spring through early summer (September to December). Biologists have come to the conclusion that the first identifiable opaque zone is formed 14 to 18 months after birth, and all subsequent opaque zones are then formed at a yearly schedule \citep{Smith2003}. Each fish in the sample was assigned an age-group based on opaque zone counts and the amount of translucent material at the margin of otolith. Age-group 0--1 comprised fish up to 18 months old ($a_{1}=18$ months) while all subsequent age-groups spanned 12 months ($a_{2} = 30$ months, $a_{3}= 42$ months, etc ...).\\

Three hazard function models were fitted to the data: a first model assumed a constant natural mortality across age-groups and throughout the period covered by the data, a common catchability and gear selectivity in estuaries and ocean (model 1, Tab.~\ref{tab:MulletModelComparison}); the second model assumed that catchability differed between esturies and ocean; and the third model assumed that both catchability and gear selectivity differed between the two habitats. The models were compared using Akaike Information Criteria (AIC) to determine which was most supported by the data \citep{Burnb03}.\\
%Sensitivity of survival analysis estimates to these data, a matrix containing 7 years and 16 age-groups, were performed by truncating the dataset in 2 ways to assess the robustness of the method to varying number of years and age-groups. The first truncation removed the last and last-two years of data to evaluate the sensitivity of parameters estimates to addition/omission of data in order to anticipate possible effects of future addition of newly available data. The second truncation removed older age-groups from 10--11 to 15--16 to evaluate the importance of few old fish on natural mortality estimates as one could think {\it a priori} that these longer-lived individuals provided a lot of information on mortality.\\


